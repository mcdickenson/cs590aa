\documentclass[12pt,letterpaper]{article}
\usepackage{amsmath,amsthm,amsfonts,amssymb,amscd}
\usepackage{fullpage}
\usepackage{lastpage}
\usepackage{enumerate}
\usepackage{fancyhdr}
\usepackage{mathrsfs}
\usepackage{mathtools}
\DeclarePairedDelimiter{\ceil}{\lceil}{\rceil}
\usepackage{xcolor}
\usepackage{tikz}
\usetikzlibrary{arrows}
\usetikzlibrary{matrix}
\usepackage[margin=3cm]{geometry}
\setlength{\parindent}{0.0in}
\setlength{\parskip}{0.05in}

\usepackage{../compsci430}


\newenvironment{answer}[1]{
  \subsubsection*{Question #1}
}


\long\def\cps590header{\begin{center}
\large\bf CPS 430/590.06 \hfill Prof.\ John Reif\\
\large\bf Design and Analysis of Algorithms \hfill Fall 2013 \\
\large\bf Homework 4 \hfill Matt Dickenson
\end{center}}

\headsep 10pt

\begin{document}

\cps590header

% DUE NOVEMBER 6 


\begin{answer}{1}
% Design a linear-time algorithm to find an odd-length cycle in a directed graph.
\end{answer}


\begin{answer}{2}
% Design an efficient algorithm to determine, in a directed graph G = (V, E), whether or not there exists a vertex v ∈ V from which all other vertices are reachable.
\end{answer}

\begin{answer}{3}
% Design an efficient algorithm to compute the number of different directed paths from vertex v to vertex w in a directed acyclic graph G.
\end{answer}

\begin{answer}{4}
% Design a linear-time algorithm to determine, in a directed acyclic graph G, whether or not there is a directed path that touches every vertex exactly once.
\end{answer}

\end{document}
