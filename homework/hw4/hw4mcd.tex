\documentclass[12pt,letterpaper]{article}
\usepackage{amsmath,amsthm,amsfonts,amssymb,amscd}
\usepackage{fullpage}
\usepackage{lastpage}
\usepackage{enumerate}
\usepackage{fancyhdr}
\usepackage{mathrsfs}
\usepackage{mathtools}
\DeclarePairedDelimiter{\ceil}{\lceil}{\rceil}
\usepackage{xcolor}
\usepackage{tikz}
\usetikzlibrary{arrows}
\usetikzlibrary{matrix}
\usepackage[margin=3cm]{geometry}
\setlength{\parindent}{0.0in}
\setlength{\parskip}{0.05in}

\usepackage{../compsci430}


\newenvironment{answer}[1]{
  \subsubsection*{Question #1}
}


\long\def\cps590header{\begin{center}
\large\bf CPS 430/590.06 \hfill Prof.\ John Reif\\
\large\bf Design and Analysis of Algorithms \hfill Fall 2013 \\
\large\bf Homework 4 \hfill Matt Dickenson
\end{center}}

\headsep 10pt

\begin{document}

\cps590header

% DUE NOVEMBER 6 


\begin{answer}{1}
% Design a linear-time algorithm to find an odd-length cycle in a directed graph.

% I am considering to do a Breadth First Search on the graph and trying to label the vertices black and white such that no two vertices labeled with the same color are adjacent.

% The reason that works is that if you label the vertices by their depth while doing BFS, then all edges connect either same labels or labels that differ by one. It's clear that if there is an edge connecting the same labels then there is an odd cycle. If not then we can color all odd labels white and all even labels black.

% O(|V|+|E|)
\end{answer}


\begin{answer}{2}
% Design an efficient algorithm to determine, in a directed graph G = (V, E), whether or not there exists a vertex v ∈ V from which all other vertices are reachable.

% For two nodes V and U, and reachability r(v), r(u), then u \in r(v) --> r(u) in r(v)
% This suggests a recursive (greedy?) algorithm should work 
% Do it like Floyd-Warshall, but just keep track of reachability
% Note that this now works even for negative paths 
% Just need to go until some r(v) = V 
\end{answer}

\begin{answer}{3}
% Design an efficient algorithm to compute the number of different directed paths from vertex v to vertex w in a directed acyclic graph G.

% input: v, w
% output: count_of_paths 

% topological ordering starting at (v)
% for each i=(2..w-1) see if there is a path (v-> ...-> i-> ...-> w) [
%   cf shortest path--we want to keep a count of all paths]
% add up the total number of paths


\end{answer}

\begin{answer}{4}
% Design a linear-time algorithm to determine, in a directed acyclic graph G, whether or not there is a directed path that touches every vertex exactly once.

% a DAG has a unique topological ordering if and only if it has a directed path containing all the vertices, in which case the ordering is the same as the order in which the vertices appear in the path

% topological order is linear

% http://en.wikipedia.org/wiki/Directed_acyclic_graph

\end{answer}

\end{document}
